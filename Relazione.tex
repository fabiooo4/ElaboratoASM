\documentclass[a4paper]{article}
\usepackage[italian]{babel}
\usepackage{float}
\usepackage{amsmath}
\usepackage{xcolor}
\usepackage{listings}
\usepackage{tikz}
\usetikzlibrary{shapes, automata, positioning}
\usepackage{circuitikz}

\usepackage{hyperref}
\hypersetup{
    colorlinks=false,
}

% Code blocks
\definecolor{codegreen}{rgb}{0,0.6,0}
\definecolor{codegray}{rgb}{0.5,0.5,0.5}
\definecolor{codepurple}{rgb}{0.58,0,0.82}
\definecolor{backcolour}{rgb}{0.95,0.95,0.95}

\lstdefinestyle{mystyle}{
	backgroundcolor=\color{backcolour},
	commentstyle=\color{codegreen},
	keywordstyle=\color{magenta},
	numberstyle=\tiny\color{codegray},
	stringstyle=\color{codepurple},
	basicstyle=\ttfamily\footnotesize,
	breakatwhitespace=false,
	breaklines=true,
	captionpos=b,
	keepspaces=true,
	numbers=left,
	numbersep=5pt,
	showspaces=false,
	showstringspaces=false,
	showtabs=false,
	tabsize=2
}

\lstset{style=mystyle}

\usepackage{color}

\definecolor{dkgreen}{rgb}{0,0.6,0}
\definecolor{gray}{rgb}{0.5,0.5,0.5}
\definecolor{mauve}{rgb}{0.58,0,0.82}

\lstset{frame=tb,
	aboveskip=3mm,
	belowskip=3mm,
	showstringspaces=false,
	columns=flexible,
	basicstyle={\small\ttfamily},
	numbers=none,
	numberstyle=\tiny\color{gray},
	keywordstyle=\color{blue},
	commentstyle=\color{dkgreen},
	stringstyle=\color{mauve},
	breaklines=true,
	breakatwhitespace=true,
	tabsize=3
}

\begin{document}
\begin{titlepage}
	\begin{center}
		\vspace*{1cm}

		\Huge
		\textbf{Elaborato ASM – Laboratorio
			Architettura degli
			Elaboratori}

		\vspace{0.5cm}
		\LARGE
		UniVR - Dipartimento di Informatica

		\vspace{1cm}
		\huge
		\textbf{Software pianificatore}
		\vspace{1.5cm}


		\vfill


		\vspace{0.8cm}

		\Large
		\textbf{Mattia Arganetto - VR502412}\\
		\textbf{Fabio Irimie - VR501504}

		\vspace{0.5cm}

		2° Semestre 2023/2024

	\end{center}
\end{titlepage}


\tableofcontents
\pagebreak

\section{Struttura del codice}
\subsection{Funzioni ausiliarie}
\subsubsection{printStr.s}
La funzione presente in questo file serve a stampare una qualsiasi stringa che finisca
con il carattere '\( \setminus 0\)'  presente nel registro \( \%ecx \).

\subsubsection{writeStrFile.s}
La funzione presente in questo file serve a scrivere una stringa in un file. I parametri
da passare alla funzione sono:
\begin{itemize}
	\item \( \%ebx \): File descriptor del file aperto
	\item \( \%ecx \): Stringa da scrivere
\end{itemize}

\subsubsection{printInt.s}
La funzione presente in questo file serve a convertire un intero presente nel registro
\( \%eax \)  in una stringa e successivamente stamparla.

\subsubsection{writeIntFile.s}
La funzione presente in questo file serve a scrivere un intero in un file. I parametri
da passare alla funzione sono:
\begin{itemize}
	\item \( \%esi \): File descriptor del file aperto
	\item \( \%eax \): Intero da scrivere
\end{itemize}

\subsubsection{atoi.s}
La funzione presente in questo file serve a convertire una stringa presente nel registro
\( \%esi \) in un intero e restituirlo nel registro \( \%eax \).

\subsubsection{bubbleSort.s}
\label{bubbleSort}
La funzione presente in questo file serve a riordinare una lista di interi presente nello
stack in base all'algoritmo scelto dall'utente:
\begin{itemize}
	\item Se viene scelto EDF (Earliest Deadline First) la lista di prodotti viene riordinata
	      in base alla scadenza crescente, oppure nel caso in cui le scadenze siano uguali, i
	      prodotti vengono riordinati in base alla priorità decrescente.

	\item Se viene scelto HPF (Highest Priority First) la lista di prodotti viene riordinata
	      in base alla priorità decrescente, oppure nel caso in cui le priorità siano uguali, i
	      prodotti vengono riordinati in base alla scadenza crescente.
\end{itemize}
I parametri da passare alla funzione sono i seguenti:
\begin{itemize}
	\item \( \%ecx \) : Offset dal primo elemento di un prodotto per indicare che valori
	      controllare per l'ordinamento:
	      \begin{itemize}
		      \item 1: Identificativo
		      \item 2: Durata
		      \item 3: Scadenza
		      \item 4: Priorità
	      \end{itemize}
	\item \( \%esi \): Offset dal primo elemento di un prodotto per indicare che valori
	      controllare nel caso in cui i valori di \( \%ecx \) siano uguali:
	      \begin{itemize}
		      \item 1: Identificativo
		      \item 2: Durata
		      \item 3: Scadenza
		      \item 4: Priorità
	      \end{itemize}
	\item \( \%edx \): Numero di valori presenti nello stack (numero di prodotti * 4)
	\item \( \%edi \): Numero di prodotti
\end{itemize}
La funzione fa affidamento ad un'altra funzione presente nel file \textbf{swapOrders.s} (\ref{swapOrders}).

\subsubsection{swapOrders.s}
\label{swapOrders}
Questa funzione serve a scambiare due prodotti presenti nello stack e viene utilizzata
dall'algoritmo di ordinamento. I parametri da passare alla funzione sono:
\begin{itemize}
	\item \( \%eax \): Indice del primo valore del primo prodotto
	\item \( \%ebx \): Indice del primo valore del secondo prodotto
	\item \( \%ebp \): Indice del primo valore del primo prodotto nello stack
\end{itemize}

\subsection{File principali}
\subsubsection{main.s}
Questo è il file principale del programma, cioè il primo che viene eseguito. All'interno
di questo file vengono ottenuti i parametri passati dalla riga di comando e successivamente
si procede con l'apertura del file passato come parametro. Se l'apertura non va a buon fine
viene stampato un errore e il programma viene terminato chiudendo così il file. Abbiamo
deciso di aprire il file una sola volta e di tenerlo aperto per tutta la durata del programma,
chiudendolo soltanto alla fine dell'esecuzione. Questo per evitare di aprire e chiudere il file
più volte.

\subsubsection{menu.s}
In questo file è presente il loop che stampa il menu ogni volta che viene completata una
pianificazione o quando il programma viene eseguito per la prima volta. L'esecuzione del
menu presuppone un file descriptor nel registro \( \%eax \) in modo da poterlo passare
poi agli altri file che necessitano di un file descriptor per leggere dal file. Ogni
volta che viene stampato il menu viene chiesto all'utente di inserire un'opzione, se
l'opzione è valida viene eseguita la funzione corrispondente, altrimenti viene stampato
un messaggio di errore e l'input viene chiesto nuovamente. Le opzioni disponibili sono:
\begin{itemize}
	\item 1: Pianifica gli ordini con l'algoritmo EDF (Earliest Deadline First)
	\item 2: Pianifica gli ordini con l'algoritmo HPF (Highest Priority First)
	\item 3: Esci dal programma
\end{itemize}

\subsubsection{plan.s}
In questo file avviene la pianificazione degli ordini. I parametri per questo file sono
da passare nello stack e sono i seguenti:
\begin{itemize}
	\item File descriptor del file aperto
	\item Algoritmo da eseguire inserito in input dall'utente
\end{itemize}
Una volta ottenuti i parametri si procede con la lettura del file, però siccome il file
rimane sempre aperto per tutta l'esecuzione del programma è necessario posizionare il
puntatore del file all'inizio del file ogni volta che si vuole fare una nuova pianifica;
questo viene fatto con la system call \textbf{lseek}:
\begin{lstlisting}[language={[x86masm]Assembler}]
  mov $19, %eax  ; syscall lseek (riposiziona il read/write offset)
  mov fd, %ebx   ; File descriptor
  mov $0, %ecx   ; Valore di offset
  mov $0, %edx   ; Posizione di riferimento (0 = inizio del file)
  int $0x80
\end{lstlisting}
La lettura del file viene effettuata leggendo un carattere alla volta inserendolo in un
buffer. Ogni volta che viene rilevato un carattere '\( \setminus n\)' si aumenta il
contatore dei prodotti.

\noindent Leggendo il file si ottengono tutti i valori dei prodotti e
vengono inseriti nello stack. Questo viene effettuato concatenando ogni intero letto
dal file all'interno di un buffer che verrà poi convertito in un intero quando viene
letto una virgola o un'andata a capo.

\noindent Una volta che tutti i prodotti sono stati inseriti nello stack si avrà di fatto
una lista di prodotti in cui ogni colonna rappresenta un parametro del prodotto (Id, Durata, Scadenza, Priorità).
Successivamente si procede con l'ordinamento della lista dei prodotti in base all'algoritmo scelto dall'utente, cioè
inserendo i parametri corretti per la funzione \textbf{bubbleSort.s} come definito
sopra (\ref{bubbleSort}).

\noindent Quando la lista è ordinata non resta che stampare in sequenza i prodotti
calcolando i tempi di inizio e la penalità per ogni prodotto. Questo viene fatto
nel file \textbf{output.s} (\ref{output}).

\noindent Dopo aver completato la pianifica vengono tolti dallo stack tutti i valori inseriti.

\subsubsection{output.s}
\label{output}
Questo file serve a stampare i prodotti in sequenza (presupponendo che la lista sia già
ordinata) e a calcolare i tempi di inizio e la penalità per ogni prodotto. I parametri
da passare a questa funzione sono:
\begin{itemize}
	\item \( \%eax \): Algoritmo scelto dall'utente
	\item \( \%ebx \): Numero di valori presenti nello stack (numero di prodotti * 4)
	\item \( \%ecx \): Numero di prodotti
\end{itemize}
Questa funzione dopo aver salvato i parametri procede a calcolare il tempo di inizio
di ogni prodotto scorrendo la lista e sommando la durata dei prodotti precedenti. Ogni
tempo di inizio viene quindi inserito nello stack.

\noindent Successivamente viene stampato l'output secondo la specifica richiesta e infine
viene calcolata la penalità totale sommando le penalità di ogni prodotto e stampandola.
Una volta completata la stampa dell'output vengono tolti tutti i tempi di inizio dallo
stack.


\section{Simulazione ordini}
\subsection{Controllo dei valori}
All'interno di ogni file di testo viene eseguito (in lettura) un controllo per verificare
la validità dei valori presenti all'interno. Se essi non rispettano le regole imposte il
programma resituisce un errore e viene interrotto.

\noindent In caso contrario il programma continua la sua esecuzione senza intoppi.

\subsection{Ordini obbligatori}
I seguenti ordini sono stati fatti per testare la correttezza del programma in situazioni
in cui la lista di ordini è corretta.

\subsubsection{EDF.txt}
Il seguente file di testo contiene una lista di ordini che devono essere organizzati in
modo tale da garantire una penalità uguale a \texttt{0} con l'algoritmo di ordinamento
\textbf{$EDF$} e una penalità maggiore di \texttt{0} con l'algoritmo di
ordinamento  \textbf{$HPF$}.

\subsubsection{Both.txt}
Il seguente file di testo contiene la lista di ordini che devono essere organizzati in
modo da garantire una penalità maggiore di \texttt{0} per entrambi gli algoritmi di
ordinamento presenti nel programma. In particolare questa lista di ordini prevede delle
scadenza più corte del singolo tempo di produzione dell'ordine in modo da riportare una
penalità elevata in entrambi gli algoritmi.

\subsubsection{None.txt}
Il seguente file di testo contiene la lista di ordini che dovranno essere organizzati in
modo da garantire una penalità uguale a \texttt{0} per entrambi gli algoritmi di
ordinamento. In questo caso le scadenze sono state impostate ad un tempo elevato in modo
da favorire il completamento degli ordini per entrambi gli algoritmi di ordinamento.

\subsection{Ordini extra}
I seguenti ordini sono stati fatti per testare la correttezza del programma in situazioni
particolari.

\subsubsection{noInput.txt}
Il seguente file di testo è vuoto e non contiene alcun ordine. Il programma restituisce
un errore e termina l'esecuzione.

\subsubsection{oneLine.txt}
Il seguente file di testo contiene un solo ordine. Il programma viene eseguito come
previsto e restituisce un output corretto.

\subsubsection{wrongInput.txt}
Il seguente file di testo contiene una stringa al posto dei valori degli ordini. Il
programma restituisce un errore e termina l'esecuzione.

\subsubsection{wrongId.txt}
Il seguente file di testo contiene un identificativo fuori dall'intervallo consentito.
Il programma restituisce un errore e termina l'esecuzione.

\subsubsection{wrongDuration.txt}
Il seguente file di testo contiene una durata fuori dall'intervallo consentito. Il
programma restituisce un errore e termina l'esecuzione.

\subsubsection{wrongDeadline.txt}
Il seguente file di testo contiene una scadenza fuori dall'intervallo consentito. Il
programma restituisce un errore e termina l'esecuzione.

\subsubsection{wrongPriority.txt}
Il seguente file di testo contiene una priorità fuori dall'intervallo consentito. Il
programma restituisce un errore e termina l'esecuzione.

\section{Scelte progettuali}
\subsection{Apertura del file}
Abbiamo deciso di aprire il file una sola volta e di tenerlo aperto per tutta la durata
del programma, chiudendolo soltanto alla fine dell'esecuzione. Questa decisione è stata
presa per risparmiare tempo e risorse, evitando di aprire e chiudere il file più volte.

\subsection{Lettura del file}
Abbiamo deciso di leggere il file ogni volta che viene richiesta una nuova pianificazione
in modo da permettere all'utente di modificare il file di input senza dover riavviare il
programma. Siccome il file è aperto solo con permessi di lettura si può modificare il
file di input tranquillamente senza errori.

\end{document}
